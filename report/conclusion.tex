\section{Conclusion}
In this report I have covered the basics of the Chan Vese image segmentation
algorithm, the DSC mesh structure and the inner workings of Maya. I also
designed a plugin for Maya that allows easy integration of any
simulation and mesh structure into the plugin itself and act as a visualizer. To
make it easier for user to compile on their own machine and to write
new simulators, I used CMaketo help create build-files on both Windows, Linux
and OS X.

The plugin is designed with two general interfaces that allow plugging any
simulator and any mesh structure into the plugin as long as they conform to the
designed interface. The mesh interface only specifies how to retrieve mesh data
from the mesh itself, and not how the mesh should behave or even how to write to
it. This means that the user can do it any way he wants. The simulator interface
specifies a way for the simulator to communicate what arguments it would like to
have, a way to pass those arguments along, a way to make the simulator advance
in time, along with a way to retrieve the mesh from the simulator. The plugin is
designed with an internal mesh storage that saves all the previous steps of the
simulator, so it can easily be browsed and rendered into an animation.

During the project I implemented a prototype of the design. It is not feature
complete and does not use the generic mesh interface since I was unable to
implement it in time. The interface itself was implemented as a header file to
make future work easier. The current prototype implements the generic
simulator interface and the internal mesh store. It allows the simulator to ask
for arguments and then return the users input on those parameters in order to
control the simulation, frame by frame.

I performed a number of tests of the plugin itself and did comparisons to the
current OpenGL solution it was meant to replace. My tests showed that the
plugin could perform all the tasks that the OpenGL solution could, and added a
number of new features such as the ability to look at previous steps of the animation
via the time line and changing simulation parameters without recompiling the
solution as long as the simulator in question tells Maya what parameters it
needs. The plugin works fine with the different Maya shaders, as long as they
do not require UV mapping.

Finally a number of fixes/improvements is proposed in Section \ref{sec:future}
that will greatly increase the usability of the plugin and aid the user
in testing their simulators/meshes or create visuals both static and animated to
help in promoting or explaining different scientific methods.


% END OF REPORT; TIME FOR APPENDIX

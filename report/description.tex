\section{Problem Description}
This section will outline the general problem that that I have attempted to
solve during the course of this project.

The goal of the project was to design and implement a piece of software that researchers in
the Image-group could use to visualize the results of a Chan-Vese segmentation.
This visualization was implemented as a plug-in for Autodesk Maya, a
professional 3D program.

The actual image segmentation is performed by a piece of software that Ulrik
Bonde wrote, which I have been granted access to. The output data from this
software is a moving tetrahedral mesh maintained by the DSC framework.
Chan-Vese and DSC is described in sections \ref{sec:chanvese} and \ref{sec:dsc}
respectively.

The current method for visualizing the results of the segmentation is a
simple OpenGL application that offers very little in the way of customization
or render options. By switching to Maya, the researchers will gain an extremely
powerful new render engine.

In order to provide the researchers with a new tool they can use, I should
implement a plugin for Maya that gives a simple GUI for retrieving the
segmentation data and tweak the segmentation itself without having to actually
leave Maya at any point.

The rest of this report assumes the reader to be at least on-par with a
bachelor in computer science or similar education.
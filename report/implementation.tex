\section{Implementation}
This section covers the implementation of the plugin, the simulator and mesh
classes as well as the CMake solution.

Because the plugin registration part only requires very few lines it have been
put in the same file as the DG node and for that reason they will also be kept
in the same subsection (subSection \ref{sec:plugnode}) of the report. Since the
design includes separate simulator- and mesh interface, these will be described
in separate subsections.

Both the plugin and the DG node is called visMesh (short for visualization mesh)
but whenever visMesh is mentioned in the report, I am talking about the DG node
specifically unless otherwise specified.

The structure of the codebase is quite simple:
\begin{table}[h]
\begin{tabular}{c c}
\begin{minipage}{7cm}
\dirtree{%
.1 SVN repository/project root.
.2 contract.
.3 Project contract.
.3 Project description.
.2 report.
.3 \LaTeX files used for the report.
.3 img.
.4 Image files used in the report.
.3 graphs.
.4 Yed graph files.
.2 visMesh.
.3 cmake.
.4 FindMaya.cmake.
.3 PLUGIN.
.4 include.
.5 ChanVese.h.
.5 gmesh.h.
.5 gsim.h.
.5 macros.h.
.5 visMesh.h.
.4 src.
.5 ChanVese.cpp.
.5 visMesh.cpp.
.4 CMakeLists.txt.
.4 visMeshSetup.mel.
.3 CMakeLists.txt.
}\end{minipage}
&
\begin{minipage}{7cm}
\texttt{FindMaya.cmake}: CMake module to find Maya, see Section \ref{sec:cmake}.\\
\texttt{ChanVese.h}: Header file defining the ChanVese simulator.\\
\texttt{gmesh.h}: Header file defining the generic mesh class.\\
\texttt{gsim.h}: Header file defining the generic simulator class.\\
\texttt{macros.h}: Header file defining convenience macros.\\
\texttt{visMesh.h}: Header file defining the plugin class.\\
\texttt{ChanVese.cpp}: Implementation of the ChanVese simulator.\\
\texttt{visMesh.cpp}: Implementation of the plugin and registration code.\\
\texttt{CMakeLists.txt}: CMake file for the project.\\
\texttt{PLUGIN/CMakeLists.txt}: CMake file for the plugin.\\
\texttt{PLUGIN/visMeshSetup.mel}: Setup script for the plugin, see Section \ref{sec:melsetup}.\\
\end{minipage}
\end{tabular}
\end{table}

\subsection{Mesh}
This subsection will explore the implementation of the mesh structures in the
project. Note that the implementation of the mesh differs from
the design. The differences and reasons behind is explained in this subsection.

A virtual class for the generic mesh was implemented in
\textit{PLUGIN/include/gmesh.h}, however due to time constraints there was not
enough time to add it to the plugin itself.

\subsubsection{Divided mesh data}
In order to replace the generic mesh a new structure was adapted. This structure
is inspired by how Maya stores mesh data. Each object consists of three lists:
\begin{enumerate*}
  \item \textit{Vertices}: This list stores all the vertices in the geometry.

  \item \textit{Face Counts}: The $n$'th entry in this list contains the number
    of vertices needed by the $n$'th face of the object.

  \item \textit{Face Connects}: Contains the order in which the vertices should
    be connected in order to form a complete face.
\end{enumerate*}

To create a mesh from the data I first read an entry in faceCounts, this entry
describes how many vertices I should get to make the specific face, I then read
that amount of entries from faceConnects, these entries are the indices of the
vertices  to get in the vertex list in order to draw the face. To draw the
next face we simple read the next entry in faceCount and read that amount of
entries from faceConnects and get the appropriate vertices from those indices.
It's important to mark how far we have moved through the faceConnects list since
we should never read the same entry twice.

To replace the generic mesh a number of smaller changes had to be made. The
biggest change was that the \texttt{std::vector<gmesh>} that should contain the
meshes was replaced by three other vectors:
\begin{lstlisting}
std::vector<MFloatPointArray> vertices;
std::vector<MIntArray> faceCounts;
std::Vector<MIntArray> faceConnects;
\end{lstlisting}
With this setup the $n$'th entry in the vectors contains the info that is needed
to create the $n$'th mesh in Maya. This implementation also means that it does
not correspond to the design laid out in Figure \ref{fig:design}, instead it
will look like Figure \ref{fig:implementation}.

\graphicc{0.8}{img/module-implementation.png}{Visualization of the
  implementation, just like Figure \ref{fig:design}, this is not a UML diagram
  but rather an illustration of the implementation.}{fig:implementation}

This implementation means that the simulator must now be responsible for
delivering the mesh data via the three functions that the mesh structure implemented
before. This means that the method \texttt{virtual gmesh* getMesh() = 0;} is
replaced with the three following methods:
\begin{lstlisting}
virtual MFloatPointArray getMeshVerts() = 0;
virtual MIntArray getMeshFaceCounts() = 0;
virtual MIntArray getMeshFaceConnects() = 0;
\end{lstlisting}

The lack of full-implementation of the mesh type does not have a significant
performance impact. The work the mesh class should have done is simply moved
from the mesh to the simulator. But some of the ease-of-customization is lost
since placing a new mesh structure in the plugin itself will require more work.

It is also worth noticing that the ability for the plugin to save meshes to files
is not implemented. Any simulator that wishes to be able to resume after Maya
have been closed should provide such functionality by itself.

\subsubsection{DSC Mesh}
There was only a little interaction with the DSC mesh in the implementation
since the Chan Vese code takes care of all the simulation/segmentation. The only
real work was the conversion from DSC to the Maya-like mesh structure.

In order to do this I created a method in the \texttt{chanVeseSim} class that
reused almost all of the code from the \texttt{draw} method already present in
the \textit{dsc\_display.h} file. The first thing to do is to strip the OpenGL
related code, four lines of code, and marking the place where it was
drawing the OpenGL polygons. In this place I inserted code that took each
triangle and instead added the vertices to the simulators internal vertex array,
and then added the needed connectivity data by appending three (since all our faces
are triangles) to the faceCount array for each face and then added a new number
to faceConnects for each vertex. When coded, the loop to save a face from the
DSC mesh looks like so:
\begin{lstlisting}
faceCounts.append(3);
for (int i = 0; i < 3; ++i) {
    V v = dsc.find_node(verts[i]).v;
    faceConnects.append(vertices.length());
    vertices.append(MFloatPoint(v[0], v[1], v[2]));
}
\end{lstlisting}

This way of saving vertices creates redundant geometry data in the vertex list
since all vertices are shared between at least three faces (any vertex in a
tetrahedron) that vertex will get saved at least three times. Instead of just
blindly adding vertices the simulator should check if it already have a vertex
on that position and add an entry with that vertex's index to the faceConnects
list. Again due to time restrictions, this was not added to the plugin.

\subsection{Simulator}
The simulator should provide a way to tell the plugin what parameters it wants,
how to advance the simulation, along with some way of retrieving the mesh for
the current state.

\subsubsection{Virtual Simulator}
Apart from the constructor and destructor the generic simulator
(\textit{include/gsim.h}) defines the following virtual functions:
\begin{itemize*}
  \item \texttt{MStringArray getArguments}: Returns a list of names for the
  parameters that the simulator wants Maya to show to the user.

  \item \texttt{void initialize}: This functions gets used to initialize the
  simulator. It takes a filepath (string) as argument, this file is the initial state of the
  simulator, which allows for ``warm starting'' it. It also takes an array of
  doubles as argument, this is the values of the arguments requested in the
  function described above.

  \item \texttt{void tick}: This function causes the simulator to advance one
  time step.

  \item \texttt{MFloatPointArray getMeshVerts}: Returns the vertices of the
  mesh in the simulator.

  \item \texttt{MIntArray getMeshFaceCounts}: Returns the face count data of
  the mesh in the simulator.

  \item \texttt{MIntArray getMeshFaceConnects}: Returns the face connectivity
  data for the mesh in the simulator.
\end{itemize*}

At this point the only type of arguments the simulator accepts are floats. This
is because I did not have time to implement a way for the simulator to tell the
plugin what datatypes it expects for the different parameters. The double
datatype was selected because it supplies precision enough for the current
simulation purposes.


\subsubsection{Chan-Vese Simulator}

Normally a Chan-Vese segmentation takes several arguments (the $\mu$, v and
$\lambda$ values) but since the current segmentation is not a correct
implementation but a dummy sphere-interpolator it only takes two arguments, the
radius of the sphere and the length of each timestep in the simulation.

The method for initializing the segmentation is copied from the
sample segmentation program provided with the segmentation library. The greatest
difference is that the simulator gets its parameters from the plugin instead of
having them hardcoded, and that an initialization \textit{*.dsc} file can be defined.
If no initialization file is specified, it will simply draw a small cube.

When the simulator class is asked to perform a step, it will check if the step size is
above 0. If it is, it will ask the segmentation to perform a step with the
specified size. If the size is 0 or below, the simulator class will write an error and return
without changing anything.


\subsection{Plugin and DG Node}
\label{sec:plugnode}
This subsection will cover the implementation of the plugin registration code
and the code that makes up the DG node.

The only two things needed to register a plugin is specific functions that need
to be exported visibly in the finished \textit{.mll} file. The two functions are
\texttt{MStatus initializePlugin(MObject obj)} and \texttt{MStatus
  uninitializePlugin(MObject obj)}.

Both take a Maya wrapper \texttt{MObject} as argument and returns a
\texttt{MStatus}. The \texttt{MObject} will become the object that Maya sees as
the plugin itself when registered, and the \texttt{MStatus} is used to tell Maya
if the plugin managed to register or deregister correctly. An example of a
failure would be if the plugin attempted to register a DG node with a node id
that is already in use.

When registering the plugin I do two things: register the plugin information
such as author, version and needed Maya API and register the DG node.  After
these things are done the \texttt{initializePlugin} function returns success to
Maya to let it know that everything should work smoothly now.

When uninitializing the plugin, the only thing done is to deregister the DG node
and tell Maya whether or not we succeeded in it.

Creating  a DG node in Maya is done by implementing a class that overrides
\texttt{MPxNode} and then overriding/implementing the following functions/static
variables:
\begin{itemize*}
  \item The constructor (can be empty).
  \item A virtual destructor (can be empty).
  \item \texttt{virtual MStatus compute (const MPlug\& plug, MDataBlock\& data)}\\
    This is called whenever a plug on the node is marked dirty and needs to be
    recomputed. This is the workhorse of the node.
  \item \texttt{static void* creator()}\\
    This is called by Maya when it wants to make an instance of the node, in my
    case the only thing it does is to return an instance of the constructor for
    the class.
  \item \texttt{static MStatus initialize()}\\
    This is called after the creator and sets up attributes for the node and
    relations between them, after this is called, the node must be able to
    function properly.
  \item \texttt{static MTypeId id}\\
    Static variable defining the ID of the node type. This must be unique, any
    clashes will result in failure at registration.
\end{itemize*}

As written above the real workhorse of the plugin is the \texttt{compute}
method. This is the method that will be called by Maya whenever the plugins
output nodes are dirty and needs to be recomputed. Every output plug is set up
with a relationship to all of the input plugs it is relaying on. The visMesh
node only have on output plug, but this plug relies on all of the input plugs,
so whenever one of the input plugs changes, the output plug gets marked as
\textit{dirty}. This allows any other nodes that ask for the output to see that
it is not ``up-to-date'' and request that the visMesh node recomputes itself.

\texttt{compute} gets called with two arguments, the \texttt{MPlug\& plug} that
should be recomputed and a \texttt{MDataBlock\& data} that contains the data for
all the nodes input plugs. The current implementation of \texttt{compute} will
first do a check to see if any of it's input plugs (excluding the time plug)
have changed, this is important since if any of them have changed, that means
the parameters for the simulator should be updated. If the plugs are all the
same (no parameters have changed), the node will check if the frame requested by
Maya is within what is currently simulated. If the data that belongs in that
frame is stored it will create a mesh object and pass it to the output plug and
mark it as \textit{clean}. If the plugs are the same, and the mesh is not in
store, visMesh will run the simulator until the requested mesh gets generated
and then pass it to the output plug. If the configuration have changed visMesh
will delete all it's current mesh data and re-initialize the simulator with the
new parameters and re-simulate until it have simulated the requested frame.

\subsection{CMake}
\label{sec:cmake}

The CMake files are responsible for telling CMake how to create a project/make
file and help make development and cross platform building easier.

In order to compile and link the DSC framework the LAPACK/BLAS module is
needed. CMake ships with a module that can find it if it's installed.
I also wished to find Maya automatically to use its header files and libraries,
as written in Section \ref{sec:platforms}. A Github user had
already created a CMake module for this, using this module getting all the
paths for Maya's libraries and include files was as simple as writing
\texttt{FIND\_PACKAGE(Maya)} in the CMake file for the plugin.

The plugin currently have two CMake files, one for the overall project and one for
the plugin itself. This will provide an easier way to add new classes to the
plugin without touching the existing code, since a new folder can be made and
added as a sub directory to the main CMake file.

The second CMake file, the one for the plugin itself, is responsible for finding
both the Maya and LAPACK paths along with the paths for the DSC project files.
I did not have time to develop a proper way of automatically detecting the DSC
project files, so instead a path to the root is given and a CMake configuration
file in the DSC project will create the needed paths. In order to set the path
the following lines are needed:
\begin{lstlisting}
SET(DSC_PROJECT_DIR "C:/SVN/BSc/dsc-repo/CODE/DSC_PROJECT")
FIND_PACKAGE(DSC_PROJECT)
\end{lstlisting}
The plugin will build on both Windows and OS X.\footnote{It have not be tested
on Linux, but a build should be possible as long as Maya, DSC, Chan-Vese etc.
are installed correctly.} The plugin will work under different Maya version,
but will need to be compiled with libraries for each specific version and with
a fairly modern compiler, for more details see \cite{Mayareq}.

\subsection{Node Setup}
\label{sec:melsetup}

Even though the plugin itself is written in C++, the plugin needs to be loaded
into Maya and the visMesh DG node needs to be placed in the scene. In order to
make this task easy, I have created a MEL script that will load the plugin,
create a transform node (so the mesh can be moved around), add a generic mesh
node that ships with Maya (to display the mesh data visMesh calculates), adds a
shader to the mesh, and create the visMesh node itself in the scene. Lastly the
script will connect the visMesh output plug to the mesh-nodes input plug and
connect the scenes time to visMeshs time input plug.

All of this could be done by hand, but this script makes the whole process
automatic in order to save the users time. The script is written below and is
commented to clarify what each line does.

\begin{lstlisting}
// Below is MEL script commands used to load the plugin.

// If the plugin is in the pluginsdir but does not autoload:
loadPlugin visMesh;

// Create a transform node so the mesh can be moved easily.
createNode transform -n visMesh1;

// Create a kMesh node that can be rendered.
createNode mesh -n visMeshShape1 -p visMesh1;

// Add a default shader to the mesh.
sets -add initialShadingGroup visMeshShape1;

// Create our actual node
createNode visMesh -n visMeshNode1;

// Connect the scenes time attribute to our input attribute.
connectAttr time1.outTime visMeshNode1.time;

// Connect the outputMesh from our node to the kMesh node.
connectAttr visMeshNode1.outputMesh visMeshShape1.inMesh;
\end{lstlisting}
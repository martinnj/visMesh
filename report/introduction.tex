\section{Introduction}

For my bachelors project I helped DIKUs Image-group by designing and
prototyping a tool to help them visualize simulations and mesh
structures. Currently their only way of visualizing their research are in a crude
OpenGL applications that are written specifically for each project.

My supervisors are currently working with image segmentation in the form of a
Chan-Vese model based segmentation algorithm and a moving mesh framework called
Deformable Simplicial Complexes (from here on just called DSC), so my project
will be centered around creating a visualization tool for those two projects
that can be adapted to fit any other project.

The implementation of the project will be in the shape of a Autodesk Maya plugin,
which will allow the user to visualize the results of their simulation and
easily provide parameters for the simulation to observe changes. These
visualizations can be used both internally to help debug an algorithm or be
rendered to images or animations that can be published.

\graphicc{0.5}{img/not_so_pretty_mix.png}{A mix of two pictures rendered in the
resulting Maya plugin. The shape is loaded from a DSC file into the simulator,
a shader have then been applied to the shape. The left image is
rendered with Mayas default software renderer, the right one is rendered with Mayas
Mental Ray renderer.}{fig:intro}

In this report I will first go over the theoretical knowledge needed to complete
the plugin The DSC framework and what Simplicial Complexes are, and then
move on to explain the Chan-Vese segmentation algorithm and how Autodesk Maya
works. Afterwards I will cover my analysis of the problem and the design of a
solution. The report will then describe the actual implementation of the plugin
and the use of CMake to make future work and usage easier.

Finally I will go over the results of the plugin and compare them to the
current OpenGL solution and go over the changes that should be done in order to
further improve the solution.

The code can be obtained by writing to either me, Kenny or Ulrik (see frontpage
for emails.)

%Fra Torbens slide:
%En indledning er:
%En kort teaser, som for den forventede læser gør rede for, hvorfor projektet er in-
%teressant, for state-of-the-art, for jeres resultater, og for forventninger til læseren.
%
%Tænk: Hvis en teknisk projektleder eller en studiekammerat skal læse indledningen
%(men ikke resten af rapporten), hvad skal vedkommende da læse for - overordnet
%- at forstå jeres projekt?
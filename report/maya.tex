\section{Autodesk Maya}
\label{sec:Maya}
The section will describe the way Autodesk Maya represents data internally and
what we can do to be able to add our own code into the program through the
plugin system.

\subsection{Internal Representation}
Maya represents all objects and operations in two internal graphs with different
characteristics and purposes, the Dependency Graph (DG) and
the Directed Acyclic Graph (DAG).

\subsubsection{Dependency Graph}

%TODO: meshes can be data in plugs, shapes are what is displayed.
The DG is a directed graph consisting of a number of nodes. Nodes in the DG can
have a number of attributes (called plugs) that can be used as either
\textit{input}, \textit{output} or \textit{unconnectable}. The
\textit{input/output} nodes define relationships between objects in Maya. All
objects and modifiers in Maya is represented in the DG as one or several nodes.

An example could be a map-displacement modifier with three input plugs and one
output plug. Input plugs would be mesh data (not a mesh object, but simply the
data needed to create one), an input displacement-map and an integer deciding
strength. The output plug will then deliver mesh data that can be used as input
to another modifier, or to render an actual mesh. Connectivity to such a node
would then be that all other nodes that needs the output mesh data connects to
the output plug and request the mesh data there. The input mesh data plug will
need to be connected to a DG node that can provide a base mesh the modifier can
work on. The integer plug can be set to be visible in the UI, so a user can
specify it through a text/numeric-input field.

In order to maintain consistency through the graph, plugs can be marked
\textit{dirty}. If, for instance, we change the strength plug of the displacement
node discussed before, the output plug will be marked dirty, and Maya will in
turn mark all plugs that read the output plug as dirty, propagating the change
across the entire DG. If any dirty plug is requested it will be recalculated. If
any of its input plugs are dirty, they will propagate the recalculation request
through the graph again. This way Maya only ever updates nodes if it has to.

There can be several reasons for plugs to become dirty, and for them to be
requested again. Plugs become dirty when they are unplugged/plugged-in (the DG
can change as nodes are added), keyframes are changed or other attributes are
modified. The most common reason a plug is called to be recalculated, is when a
renderer calls for the final result of the DG.

\subsubsection{Directed Acyclic Graph}

Whereas the DG nodes usually is objects and modifiers in the Maya scenes, the
DAG nodes are transformation nodes, locator nodes and similar unrenderable
operations. DAG nodes specify operations to be performed inside the 3D space
such as translation and rotation. DAG nodes are usually set as parents of DG
nodes but these connections are not normally apparent.

\graphicc{1}{img/dagdg-example.png}{An example scene in Maya with a single
  object, a number of modifiers and dataflow- and relationship arrows.}{fig:dagdg}

Figure \ref{fig:dagdg} shows an example scene in Maya, where a DG node that
generates geometry based on user input provides a mesh for a surface noise
generator. The generator outputs a mesh that the viewport, production renderers,
or any other DG node that wishes so, can pull. Scale- and rotation transformation
modifiers have been provided. The solid arrows represent data flowing through
the graph, the dashed ones represents relationships; a parent will point to its
child.

\subsection{Plugin System}
\label{sec:pluginsystem}
Maya allows for several different forms of customization:
\begin{itemize}
  \item MEL-scripts (Maya Extended Langauge)
  \item Python-scripts (interpreted within Maya itself.)
  \item External python plugins through Mayas own modules.
  \item C++ Maya plugins compiled and linked as dynamic libraries.
\end{itemize}

I will be writing a plugin using C++. This language was chosen for three reasons:
both the DSC and Chan-Vese codebase I will be integrating with is written in
C++, and the people that will use the plugin already have experience with C++
which makes it easier to take over once my project have ended. Finally C++
can be an extremely efficient language if written properly, providing only a
minimum of overhead for the link between simulator and plugin.

A C++ plugin gets compiled to a dynamic library, which on windows is called
\textit{.mll} (this is actually a DLL file) and have two exported entry points
called \texttt{initializePlugin} and \texttt{uninitializePlugin} which Maya will
call when registering and deregistering the plugin respectively.